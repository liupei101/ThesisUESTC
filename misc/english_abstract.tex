
\begin{englishabstract}
	Survival analysis, aiming at predicting the probability that individual experience the interested event at the observation, utilizes survival predictive model to find the relation between predictors and the interested event, or explore the import impact factor and pattern of the interested event. It is an important tool for survival risk prediction in healthcare, finance, industry and other fields. Particularly in the clinical disease research, survival analysis methods are much essential in many applications.
	
	The traditional survival analysis methods, such as Cox proportional hazard model and first hitting time model, solved the problem of using the model to predict individual survival risk or probability. This sort of traditional survival analysis methods is simple, intuitive and interpretable at many scenarios. However, the prior assumption and linear essence in them restrict the prediction performance at some extend. To solve the problem, researchers proposed survival analysis methods that utilized tree model or others as its base learner to enhance the model performance. Although it weakened the prior assumptions and overcame the requirement of variable independence, the model performance was limited as it did not abandon the assumptions totally. Moreover, the deep learning based survival predictive models were also proposed, which exploited complex neural network to represent the latent relation between predictors and the interested event. Although the models have powerful representation function, they also sacrifice the interpretability of the models, and then restrict its applications in the real world.

    In order to solve the problems, based on the gradient boosting decision tree, this paper proposed two novel survival analysis method named HitBoost and BecCox. The HitBoost method utilizes multi output gradient boosting decision tree to directly predict the density function of the first hitting time. In addition to using the likelihood function as the objective function, the concordance index approximated by the convex function is also added as a part of it. The HitBoost method is no longer based on any prior assumptions and still has a certain model interpretability. The results on datasets shown its goodness on model prediction. The BecCox method utilizes the single output gradient boosting decision tree to predict the proportional hazard of the interested event occurrence. The models of the Cox genre utilize tree model as its base learner, take the partial likelihood function as the objective function and use the gradient boosting algorithm to optimize the model. However, if there are many event occurrences in the dataset, the model would suffer from a decreased accuracy as the partial likelihood function is not enough precise. At the same time, such traditional tree models are prone to overfitting due to the lack of regularization term. In the objective function of BecCox, we utilize a more precise partial likelihood function, combined with a concordance index approximated by the convex function, which indeed narrows the prediction bias brought by the objective function.

	We first deduced the gradients of the custom objective function with respect to the predicted value, and then implemented the HitBoost and BecCox methods under XGBoost framework according to the different types of the output. The experimental results on the four benchmarks show that the HitBoost method is more accurate in predicting individual survival functions than the method that follows the prior assumption and the commonly used Random Survival Forest; the BecCox method has better risk prediction performance than the classic Cox proportional hazard model, and the other commonly used machine learning method followed the proportional hazard assumption. Finally, we used the clinical data of breast cancer patients from the Breast Cancer Research Center of West China Hospital of Sichuan University. After the data was cleaned and screened, the proposed HitBoost method was used to build a recurrence prognosis model for the early stage breast cancer patients. At the same time, we also shown the applications of the recurrence prognosis model, such as exploring important factors that affect breast cancer recurrence and making treatment recommendations. Therefore, the gradient boosting decision tree based methods proposed in this paper can be utilized as an effective survival analysis method for the research and application of clinical diseases or other specific interested events.

	\englishkeyword{survival analysis, machine learning, gradient boosting decision tree, proportional hazard model, disease prognosis research}
\end{englishabstract}


