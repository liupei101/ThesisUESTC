
\begin{englishabstract}
	Survival analysis, aiming at predicting the probability that individuals experience the interested event at the observation, utilizes survival predictive model to find the relation between predictors and the interested event, or explore the important impact factor and pattern of occurrence of the interested event. It is an important tool for risk estimation in healthcare, finance, and other fields. Particularly in clinical disease research, survival analysis methods are essential in many scenarios.
	
	The traditional survival analysis methods, such as the Cox proportional hazard model and the first hitting time model, solved the problem of using the model to predict individual survival probability. This sort of traditional survival analysis methods is simple, intuitive and interpretable. However, the prior assumption and linear essence in them restrict the prediction performance to some extend. To solve the problem, researchers proposed survival analysis methods that utilized tree model or others as its base learner to enhance the model performance. Although it weakened the prior assumptions and overcame the requirement of variables independence, the model performance was limited as it did not abandon the assumptions totally. When the distribution of survival data is unknown or lacks prior knowledge, these assumptions will limit the predictive performance of the model. Moreover, the deep learning based survival predictive models were also proposed. Although the models have powerful representation function, they also sacrifice the interpretability of the models and then restrict their applications in the real world.

    When the survival data passes the hypothesis test of proportional hazard or has prior knowledge, many algorithms of the Cox genre based on the Cox proportional hazard model have a wide range of application scenarios. Among them, the Cox genre algorithms built with the tree model as the base learner, with the advantages of the non-linear model, enhance the representation of the model. They often take the partial likelihood function as the objective function and use the gradient boosting algorithm to optimize the model structure. However, if there are many event occurrences in the dataset, the model will suffer from parameters estimation bias and a decreased accuracy as the partial likelihood function is not enough precise. At the same time, such traditional tree models are prone to overfitting due to the lack of regularization terms.

    In order to solve the problems, based on the gradient boosting decision tree, this paper proposed two novel survival analysis methods named HitBoost and BecCox. The HitBoost method utilizes multi output gradient boosting decision tree to directly predict the density function of the first hitting time. In addition to using the likelihood function as the objective function, the concordance index approximated by the convex function is also added as a part of it. The HitBoost method is no longer based on any prior assumptions and still has a certain model interpretability. And the BecCox method, which mainly optimizes a series of Cox genre algorithms, utilizes the single gradient boosting decision tree to predict the proportional hazard of the interested event occurrence. In the objective function of BecCox, we utilize a more precise partial likelihood function, combined with a concordance index approximated by the convex function, which indeed narrows the prediction bias brought by the objective function.

	The paper first deduced the gradients of the custom objective function with respect to the predicted value, and then implemented the HitBoost and BecCox methods under XGBoost framework according to the different types of the output. The experimental results on the four benchmarks show that the HitBoost method is more accurate in predicting individual survival functions than the method that follows the prior assumption and the commonly used Random Survival Forest; the BecCox method has better risk prediction performance than the classic Cox proportional hazard model, and the other Cox genre algorithms. Finally, we used the clinical data of breast cancer patients from the Breast Cancer Research Center of West China Hospital of Sichuan University. The proposed HitBoost method was utilized to build a recurrence prognosis model for the early stage breast cancer patients. At the same time, with the advantages of the HitBoost method, the paper also showed the applications of the breast cancer recurrence prognosis model, such as exploring important factors that affect breast cancer recurrence and making treatment recommendations. Therefore, the gradient boosting decision tree based optimization methods proposed in this paper can be utilized as an effective survival analysis method for the research and application of clinical diseases or other specific interested events.

	\englishkeyword{survival analysis, machine learning, gradient boosting decision tree, proportional hazard model, risk prediction}
\end{englishabstract}


