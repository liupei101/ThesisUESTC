
\begin{englishabstract}
    Survival analysis (a.k.a. time-to-event analysis) has a wide range of applications in healthcare, finance, and other fields. Particularly in clinical disease research, survival analysis plays an important role. It aims to study the probability that individuals experience the interested event during the observation, find the relationship between the predictors and the outcome of interested event, and thus explore the important factor and the pattern of interested event occurrence. Survival analysis methods mainly exploit the model to learn the relationship between the observed variables and the distribution of time-to-event from available data to achieve this.
	
	However, the previous survival analysis methods have some shortcomings. First of all, in the terms of model assumption, some statistical linear models and ensemble tree models assume the distribution function of time-to-event as a specific expression with parameters. When the distribution of data is unknown or a priori knowledge is lacking, these assumptions will largely limit the prediction accuracy of the model. Secondly, from the view of model interpretability, although some deep learning based models are of powerful representation ability, they cannot convincingly explain the impact of the observed variables and then restrict their applications in the real world. Finally, when the distribution of data is known or a priori knowledge is sufficient, many algorithms of the Cox genre based on the Cox proportional hazard model take the partial likelihood estimation function as the objective function. However, if there are many event occurrences in the survival data, the model will suffer from parameters estimation bias and a decreased accuracy as the partial likelihood estimation function is not precise enough. At the same time, some ensemble tree models of the Cox genre are prone to overfitting due to the lack of regularization terms.

    In order to solve these problems, based on the gradient boosting decision tree, this thesis studied and proposed two novel survival analysis methods and applied them to the prognosis of breast cancer. Specifically, the main content and distributions of this thesis are summarized as follows:

    (1) This thesis proposed a method named HitBoost. The HitBoost method utilizes multi-output gradient boosting decision tree to directly predict the density function of the first hitting time (time-to-event). In addition to taking the maximum likelihood estimation function as the objective function, the concordance index approximated by a convex function is also added as a part of it. The HitBoost method is no longer based on any prior assumptions and still has a certain model interpretability.
    
    (2) This thesis proposed a method named BecCox. The BecCox method, which mainly optimizes the algorithm of the Cox genre, utilizes a single gradient boosting decision tree to predict the proportional hazards of interested event occurrence. In objective function of BecCox, this thesis exploits a more precise partial likelihood estimation function, combined with a concordance index approximated by a convex function, to narrow the prediction bias brought by the inappropriate objective function.
    
    (3) This thesis used the clinical data of breast cancer patients from the Breast Cancer Research Center of West China Hospital of Sichuan University and applied the proposed survival analysis method to build a recurrence prognosis model for the early stage breast cancer patients. At the same time, this thesis also showed the applications of the breast cancer recurrence prognosis model, such as exploring important factors that affect breast cancer recurrence and making treatment recommendations.

	To implement the proposed methods, this thesis first deduced the gradients of the custom objective function with respect to the predicted value, and then realized the model training according to the different types of model output under the XGBoost framework. The experimental results on four benchmarks (i.e. WHAS, SUPPORT, METABRIC and ROTT2) showed that the HitBoost method is with concordance index of 0.929190, 0.631281, 0.668679 and 0.705427, respectively. It improved the performance of the second-best method with the highest value of about 2.8\%, and also surpassed methods that follow the prior assumptions and the commonly used Random Survival Forest. And for the BecCox method, it is with concordance index of 0.898320, 0.631837, 0.645986, 0.702102, respectively. BecCox increased the metrics of the second-best method with the highest value of about 1.7\%, and also surpassed the classic Cox proportional hazards model and the other algorithms of Cox genre. Therefore, the gradient boosting decision tree based optimization methods proposed in this thesis can be utilized as an effective survival analysis method for the research of clinical diseases or other specific interested events.

	\englishkeyword{survival analysis, machine learning, gradient boosting decision tree, proportional hazard model, risk prediction}
\end{englishabstract}


