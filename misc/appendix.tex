
\thesisappendix

\chapter{生存分析优化方法源代码}

基于本文所述的HitBoost和BecCox模型框架、自定义目标函数、目标函数梯度定理,我们在可拓展的XGBoost梯度提升树框架下,根据模型框架的不同类型分别实现了提出的生存分析优化方法。源代码可以参考:
\begin{itemize}
  \item HitBoost:\emph{https://github.com/liupei101/libsurv/tree/master/libsurv/hitboost};
  \item BecCox:\emph{https://github.com/liupei101/libsurv/tree/master/libsurv/ciboost}。
\end{itemize}
源代码中核心的目标函数梯度计算部分,主要基于numpy库,使用了向量化技巧来加速目标函数梯度的计算,从而大幅度缩短模型训练的时间。

源代码提供的libsurv包中,我们实现了常用的生存预测模型建立、评估、分析流程。首先,在生存数据读取方面,我们提供了统一的生存数据构建函数,方便各个生存预测模型读取训练数据。同时,包还提供了读取公开数据集WHAS、METABRIC和模拟数据集的接口。其次,在模型训练方面,我们提供了统一的模型训练接口,实现了查看模型学习曲线的功能。最后,在模型评估方面,一致性指数计算函数在各个模型类的方法中均有实现。此外,libsurv包\emph{https://github.com/liupei101/libsurv}提供:
\begin{itemize}
  \item 贝叶斯超参数搜索工具,用于搜索模型的最佳超参数;
  \item 方法使用示例,展示HitBoost和BecCox模型从建立到评估的整体流程。
\end{itemize}