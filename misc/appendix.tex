
\thesisappendix

\chapter{梯度计算相关定理的证明}

\section{HitBoost梯度计算}

\subsection{目标函数$L_1$}

下面给出对定理\ref{thm:1.1}和\ref{thm:1.2}的证明:
\begin{proof}
已知$$L_1 = -\sum_{i=1}^{n} \left[ I(\delta_i=1)\cdot ln(\hat{y}_{T_i}^i) + I(\delta_i=0)\cdot ln(1-\sum_{t\le T_i}\hat{y}_t^i) \right]$$ 对观测时间和观测事件状态分别为$\delta_k$和$T_k$的个体$k$,如果$\delta_k = 1$,那么$L_1$中和个体$k$有关的部分为$ln(\hat{y}_{T_k}^k)$,所以$L_1$关于$\hat{y}_t^k$的一阶梯度为$$\frac{\partial L_1}{\partial \hat{y}_t^k} \mid_{\delta_k = 1} = I(t=T_k)\cdot \frac{-1}{\hat{y}_t^k}$$ 否则,如果$\delta_k = 0$,那么$L_1$中和个体$k$有关的部分为$ln(1-\sum_{t\le T_k}\hat{y}_t^k)$,对其求导数,我们可以得到相应的梯度表达式:$$\frac{\partial L_1}{\partial \hat{y}_t^k} \mid_{\delta_k = 0} = I(t\le T_k)\cdot \frac{1}{1-\hat{F}(k, T_k)}$$ 其中累积风险函数$\hat{F}$用来简化表达式。综上所证,如定理\ref{thm:1.1}所示,我们有如下$L_1$关于$\hat{y}_t^k$的\textbf{一阶梯度}$$
\frac{\partial L_1}{\partial \hat{y}_t^k}=
\begin{cases}
  I(t=T_k)\cdot \frac{-1}{\hat{y}_t^k} & \text{if } \delta_k = 1,\\
  I(t\le T_k)\cdot \frac{1}{1-\hat{F}(k, T_k)} & \text{if } \delta_k = 0.
\end{cases}
$$ 不需要太多的推导,我们可以直接在定理\ref{thm:1.1}的基础上,直接对上面的表达式求导得到$L_1$的\textbf{二阶梯度},即定理\ref{thm:1.2}如下:$$
\frac{\partial^2 L_1}{\partial \hat{y}_t^k}=
\begin{cases}
  I(t=T_k)\cdot \frac{-1}{{(\hat{y}_t^k)}^2} & \text{if } \delta_k = 1,\\
  I(t\le T_k)\cdot \frac{1}{{[1-\hat{F}(k, T_k)]}^2} & \text{if } \delta_k = 0.
\end{cases}
$$
\end{proof}

\subsection{目标函数$L_2$}

为了方便推导$L_2$的梯度,我们分别考虑其分子$\beta$和分母$\alpha$。下面给出对定理\ref{thm:1.3}和\ref{thm:1.4}的证明:
\begin{proof}
已知$$\alpha = \sum_{(i,j)\in \Omega} W_{i,j}$$ 和 $$W_{i,j} = -\left[ \hat{F}(i, T_i) - \hat{F}(j, T_i) \right]$$ 对观测时间和观测事件状态分别为$\delta_k$和$T_k$的个体$k$,我们需要推导$\alpha$关于$\hat{y}_t^k$ 的梯度。

首先,我们考虑集合$\Omega$,从中找到和个体$k$有关的元素。遍历集合$\Omega$的所有元素,和个体$k$相关的元素可以被归纳到两个不相交的子集中:$$\Omega_1=\{(k,i) \mid \delta_k=1,T_k < T_i\}$$ 和 $$\Omega_2=\{(i,k) \mid \delta_i=1,T_i < T_k\}$$ 从集合意义上来说,$\Omega_1$表示实际风险小于$k$的个体和$k$组成的元组的集合,$\Omega_2$表示实际风险大于$k$的个体和$k$组成的元组的集合。

如果$\delta_k = 1$,那么$\alpha$中和个体$k$有关的部分可以展开为$$\alpha \mid_{\delta_k=1}=\sum_{(k,i)\in \Omega_1} -\left[ \hat{F}(k, T_k) - \hat{F}(i, T_k) \right] + \sum_{(i,k)\in \Omega_2} -\left[ \hat{F}(i, T_i) - \hat{F}(k, T_i) \right] $$ 我们首先考虑集合$\Omega_1$,定义$\alpha_1$为:\[
\begin{split}
\alpha_1 &= \sum_{(k,i)\in \Omega_1} -\left[ \hat{F}(k, T_k) - \hat{F}(i, T_k) \right] \\
         &= \sum_{(k,i)\in \Omega_1} - ( \sum_{\tau \le T_k} \hat{y}_{\tau}^i - \sum_{\tau \le T_k} \hat{y}_{\tau}^k )
\end{split}
\] 所以,$\alpha_1$关于$\hat{y}_t^k$的一阶梯度是$$\frac{\partial \alpha_1}{\partial \hat{y}_t^k} = I(t\le T_k)\cdot {\sum\limits_{i: T_i>T_k}(-1)}$$ 同理,考虑集合$\Omega_2$,定义$\alpha_2$为:\[
\begin{split}
\alpha_2 &= \sum_{(i,k)\in \Omega_2} -\left[ \hat{F}(i, T_i) - \hat{F}(k, T_i) \right] \\
         &= \sum_{(i,k)\in \Omega_2} - ( \sum_{\tau \le T_i} \hat{y}_{\tau}^k - \sum_{\tau \le T_i} \hat{y}_{\tau}^i )
\end{split}
\] 所以,$\alpha_2$关于$\hat{y}_t^k$的一阶梯度是$$\frac{\partial \alpha_2}{\partial \hat{y}_t^k} = \sum\limits_{i: \delta_i=1,T_i<T_k} I(t\le T_i)$$ 故我们可以总结$\alpha$关于$\hat{y}_t^k$的一阶梯度为$$\frac{\partial \alpha}{\partial \hat{y}_t^k} = \frac{\partial \alpha_1}{\partial \hat{y}_t^k} + \frac{\partial \alpha_2}{\partial \hat{y}_t^k}$$ 否则,如果$\delta_k = 0$,则和个体$k$有关的子集只剩$\Omega_2$。所以,我们可以证明得到定理\ref{thm:1.3}给出的\textbf{一阶梯度}表达式:$$
\frac{\partial \alpha}{\partial \hat{y}_t^k}=\alpha^{'}=
\begin{cases}
I(t\le T_k)\cdot {\sum\limits_{i: T_i>T_k}(-1)} + \sum\limits_{i: \delta_i=1,T_i<T_k} I(t\le T_i) & \text{if } \delta_k = 1,\\
\sum\limits_{i: \delta_i=1,T_i<T_k} I(t\le T_i) & \text{if } \delta_k = 0.
\end{cases}
$$ 由于上述的一阶梯度表达式中没有出现$\hat{y}_t^k$,所以$\alpha$关于$\hat{y}_t^k$的\textbf{二阶梯度}为:$$
\frac{\partial^2 \alpha}{\partial \hat{y}_t^k}=\alpha^{''}=0
$$

对于$L_2$项的分子$\beta$,已知$$\beta = \sum_{(i,j)\in \Omega} -\left[\hat{F}(i, T_i) - \hat{F}(j, T_i)\right] \cdot \phi\left[ \hat{F}(i, T_i), \hat{F}(j, T_i) \right]$$ 对于个体$k$,现在我们需要推导$\beta$关于$\hat{y}_t^k$的梯度。和分母$\alpha$的表达式不同,分子$\beta$中含有函数$\phi(\cdot)$。同样地,我们需要考虑集合$\Omega$中和个体$k$相关的元素,将其分为两个不相交的子集$\Omega_1$和$\Omega_2$。

如果$\delta_k = 1$,那么$\beta$中和个体$k$有关的部分可以展开为$$\beta \mid_{\delta_k=1}=\sum_{(k,i)\in \Omega_1} W_{k,i}\cdot \phi\left[ \hat{F}(k, T_k), \hat{F}(i, T_k) \right] + \sum_{(i,k)\in \Omega_2} W_{i,k}\cdot \phi\left[ \hat{F}(i, T_i), \hat{F}(k, T_i) \right] $$ 首先考虑上式加号左边的部分,定义$\beta_1$为$$\beta_1 = \sum_{(k,i)\in \Omega_1} W_{k,i}\cdot \phi\left[ \hat{F}(k, T_k), \hat{F}(i, T_k) \right]$$ 由于当$\hat{F}(k, T_k) - \hat{F}(i, T_k) \ge \gamma$时,函数$\phi(\cdot)=0$,所以我们可以简化$\beta_1$为$$\beta_1 = \sum_{(k,i)\in \Omega_1} I(-W_{k,i} < \gamma) \cdot W_{k,i}\cdot [W_{k,i} + \gamma]^n $$ 从上面的对$\alpha$梯度的推导我们已经得到$W_{k,i}$关于$\hat{y}_t^k$的梯度为$$\frac{\partial W_{k,i}}{\partial \hat{y}_t^k} = I(t\le T_k)\cdot {\sum\limits_{i: T_i>T_k}(-1)}$$ 所以我们借助链式法则,可以推导得到$\beta_1$关于$\hat{y}_t^k$的一阶梯度如下$$
\frac{\partial \beta}{\partial \hat{y}_t^k} \mid_{\Omega_1} = I(t\le T_k)\cdot \sum\limits_{(k,i)\in \Omega_1} {I(-W_{k,i}<\gamma)\cdot (W_{k,i}+\gamma)^{n-1}\cdot [-(n+1)\cdot W_{k,i}-\gamma]}
$$ 同理,对于$\beta$展开式的右边部分$\beta_2=\sum_{(i,k)\in \Omega_2} W_{i,k}\cdot \phi\left[ \hat{F}(i, T_i), \hat{F}(k, T_i) \right]$,我们可以得到其一阶梯度如下$$
\frac{\partial \beta}{\partial \hat{y}_t^k} \mid_{\Omega_2} = \sum\limits_{(i,k)\in \Omega_2} {I(t\le T_i)\cdot I(-W_{i,k}<\gamma)\cdot (W_{i,k}+\gamma)^{n-1}\cdot [(n+1)\cdot W_{i,k}+\gamma]}
$$ 否则,如果$\delta_k = 0$,则$\beta$的表达式中和个体$k$有关的子集只剩$\Omega_2$,所以此时$\beta$关于$\hat{y}_t^k$的一阶梯度和$\beta_2$一样。最终,我们可以得到定理\ref{thm:1.3}给出的\textbf{一阶梯度}表达式:$$
\frac{\partial \beta}{\partial \hat{y}_t^k}=\beta^{'}=
\begin{cases}
\frac{\partial \beta}{\partial \hat{y}_t^k} \mid_{\Omega_1} + \frac{\partial \beta}{\partial \hat{y}_t^k} \mid_{\Omega_2} & \text{if } \delta_k = 1,\\
\frac{\partial \beta}{\partial \hat{y}_t^k} \mid_{\Omega_2} & \text{if } \delta_k = 0.
\end{cases}
$$

有了一阶梯度$\beta^{'}$的表达式,我们可以通过对$\beta^{'}$求导来推导$\beta$的二阶梯度。在推导二阶梯度的过程中,由于仅涉及梯度求解的链式法则,所以我们直接给出对$\beta^{'}$求导的结果,即定理\ref{thm:1.4}:\[
\begin{split}
\frac{\partial^2 \beta}{\partial \hat{y}_t^k} \mid_{\Omega_1} =& I(t\le T_k)\cdot \sum\limits_{(k,i)\in \Omega_1} I(-W_{k,i}<\gamma)\cdot \\
  & \left\{(n+1)\cdot (W_{k,i}+\gamma)^{n-1} + (n-1)\cdot (W_{k,i}+\gamma)^{n-2}\cdot [(n+1)\cdot W_{k,i}+\gamma]\right\} \\
\frac{\partial^2 \beta}{\partial \hat{y}_t^k} \mid_{\Omega_2} =& \sum\limits_{(i,k)\in \Omega_2} I(t\le T_i)\cdot I(-W_{i,k}<\gamma)\cdot \\
  & \left\{(n+1)\cdot (W_{i,k}+\gamma)^{n-1} + (n-1)\cdot (W_{i,k}+\gamma)^{n-2}\cdot [(n+1)\cdot W_{i,k}+\gamma]\right\}
\end{split}
\] 且有$$
\frac{\partial^2 \beta}{\partial \hat{y}_t^k}=\beta^{''}=
\begin{cases}
\frac{\partial^2 \beta}{\partial \hat{y}_t^k} \mid_{\Omega_1} + \frac{\partial^2 \beta}{\partial \hat{y}_t^k} \mid_{\Omega_2} & \text{if } \delta_k = 1,\\
\frac{\partial^2 \beta}{\partial \hat{y}_t^k} \mid_{\Omega_2} & \text{if } \delta_k = 0.
\end{cases}
$$
\end{proof}

\section{BecCox梯度计算}

\subsection{目标函数$L_1$}
...

\subsection{目标函数$L_2$}
...
