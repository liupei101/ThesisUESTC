	
\begin{chineseabstract}
    生存分析,旨在预测个体在观测期发生某个特定事件的概率。其通过建立生存预测模型以寻找感兴趣的事件与预测变量之间复杂的因果关系,或者探究感兴趣事件的重要影响因子及其影响模式,是医疗健康、金融、工业等领域生存风险估计的重要工具。特别是在临床疾病预后研究中,生存分析方法发挥着必不可少的作用。

    传统的统计生存分析方法,如Cox比例风险模型和FHT(First Hitting Time)首次命中时间模型,解决了利用模型预测个体发生特定事件风险或概率的问题。这类传统的生存分析方法简单直观,具有很好的模型解释性。但它们大多为基于某种先验假设的线性模型,这很大程度上限制了这一类统计模型的预测性能。为此,研究者提出采用树模型或其他模型作为基学习器的生存分析算法,这些算法较大地提升了模型预测性能。虽然它们弱化了先验假设,克服了观测变量独立性要求,但还是像传统方法一样无法完全抛弃,这在一定程度上限制了模型性能。此外,基于深度学习的生存分析模型也被提出,其使用复杂的神经网络来捕捉预测变量和感兴趣事件之间的潜在关系。尽管这类模型拥有强大的表达能力,但是也牺牲了模型的解释性,限制了它们在实际应用中的发展。

    针对这些问题,基于梯度提升树,本文系统地研究并提出了两种新的生存分析方法:HitBoost方法和BecCox方法。HitBoost方法使用多输出的梯度提升树,直接预测感兴趣事件发生时间的概率分布。其除了使用极大似然估计函数作为目标优化函数外,还引入凸函数近似的一致性指数作为优化函数。HitBoost不再基于任何先验假设,但它仍然具有一定的可解释性,在多个数据集上测试表明其模型性能优异。BecCox方法使用单输出的梯度提升树来预测感兴趣的事件发生的风险比例。以树模型作为基学习器构建的Cox流派系列算法,通常在优化偏似然估计函数的策略下使用梯度提升算法来优化模型。但是在数据中发生大量事件的情况下,偏似然估计函数的近似程度会影响模型的优化,同时模型会由于缺少正则化项而容易出现过拟合。而BecCox方法在目标函数上,使用相比之前更加精确的偏似然估计函数,并且添加凸函数近似的一致性指数来缩小目标函数给模型预测带来的偏差。

    在本论文中,我们首先推导了自定义的目标函数关于模型预测值的梯度,然后根据预测目标的不同类型借助XGBoost框架分别实现了HitBoost和BecCox算法。在四个公开数据集上的实验结果显示,HitBoost方法相比遵循先验假设的方法以及常用的随机生存森林,对个体风险函数的预测更加准确;BecCox方法相比经典的Cox比例风险模型,以及其它常用的基于比例风险假设的机器学习方法,也有着更好的风险预测性能。最后,我们使用来自四川大学华西医院乳腺临床研究中心的乳腺癌患者临床数据,将提出的HitBoost方法应用于构建早期乳腺癌患者复发预后模型,并借助预后模型探究影响乳腺癌复发的重要因子和实施治疗推荐。本文提出的基于梯度提升树的两种生存优化方法可以作为一种有效的生存分析方法用于疾病或其它特定事件的预测研究和应用。


    \chinesekeyword{生存分析,机器学习,梯度提升树,比例风险模型,疾病预后研究}
\end{chineseabstract}

