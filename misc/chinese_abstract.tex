	
\begin{chineseabstract}
生存分析,作为一个旨在预测个体在观测期发生某个特定事件概率的研究方向,在医疗健康和金融等领域有着重要的应用。传统的基于统计学的生存分析方法,如Cox比例风险模型,解决了利用模型预测个体发生特定事件风险或概率的问题,但它们大多为基于某种先验假设的线性模型,这很大程度上限制了这一类统计模型的预测性能。近些年来,由于机器学习方法和技术的迅速发展,传统的生存分析方法在数据处理能力和预测性能等方面都得到了极大的提升。这一类方法基于Cox比例风险假设,以树模型或其他非线性模型作为基学习器,在优化偏似然估计函数的策略下使用梯度提升算法来优化模型,但是在发生大量事件的情况下,偏似然估计函数的近似程度会影响模型的优化,同时这类传统的树模型由于缺少正则化项而容易出现过拟合。为了解决这些问题,我们采用更加精确的偏似然估计函数作为主要优化目标,同时加入凸函数近似的一致性指数来调整个体预测风险排序,在推导目标函数高阶梯度后结合梯度提升树框架XGBoost实现了基于Cox比例风险假设的生存分析方法——BecCox。在四个公开数据集上的实验结果显示,BecCox生存分析方法相比经典的Cox比例风险模型,以及其它常用的基于比例风险假设的机器学习方法,有着更好的风险预测性能。同时,我们使用来自四川大学华西医院乳腺临床研究中心的乳腺癌患者临床数据,将提出的BecCox方法应用于建立乳腺癌预后复发风险预测模型,实验结果显示该预测模型有着较好的风险区分度,可以在乳腺癌诊断中达到辅助诊疗的目的。所以,本文提出的基于梯度提升树的BecCox方法可以作为一种有效的生存分析方法用于疾病或其它特定事件的预测研究。

\chinesekeyword{生存分析,机器学习,梯度提升树,风险比例模型,风险预测,乳腺癌预后}
\end{chineseabstract}

