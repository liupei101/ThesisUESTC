	
\begin{chineseabstract}
    生存分析,旨在预测个体在观测期发生感兴趣事件的概率。它通过建立生存预测模型来寻找观测变量与感兴趣事件之间潜在的因果关系,或探究感兴趣事件的重要影响因子及其影响模式,是医疗健康、金融等领域进行风险估计的重要工具。特别是在临床疾病预后研究中,生存分析方法发挥着重要作用。

    传统的统计生存分析方法,如经典的Cox比例风险模型和FHT首次命中时间模型,解决了利用模型预测个体发生感兴趣事件概率的问题。这类传统的生存分析方法简单直观,具有很好的模型解释性。但它们大多为基于某种先验假设的线性模型,这在很大程度上限制其预测性能。为此,研究者提出采用树模型或其他模型作为基学习器的生存分析算法,较大地提升了模型预测性能。虽然它们弱化了先验假设,克服了观测变量独立性要求,但还是像传统方法一样无法完全抛弃。当生存数据分布未知或缺乏先验知识时,这些假设会限制模型的预测性能。此外,基于深度学习的生存分析模型也被提出。尽管这类模型拥有强大的表达能力,但是也牺牲了模型的解释性,限制了它们在各个领域的应用。

    当生存数据通过了比例风险假设检验或具备先验知识时,基于Cox比例风险模型的Cox流派系列算法有着相当广泛的应用场景。其中,以树模型作为基学习器构建的Cox流派系列算法,凭借非线性模型的优势,增强了模型的表达能力。它们往往将偏似然估计函数作为优化目标,使用梯度提升算法来优化模型结构。但是当数据中有大量事件发生时,偏似然估计函数的近似程度会影响模型的参数估计,导致模型的预测性能下降。同时,这些树模型会由于缺少正则化项而容易出现过拟合。

    针对这些问题,基于梯度提升树,本文系统地研究并提出了两种新的生存分析方法:HitBoost方法和BecCox方法。HitBoost使用多输出的梯度提升树,直接预测发生感兴趣事件时间的概率分布。它除了使用极大似然估计函数作为目标优化函数外,还引入凸函数近似的一致性指数作为优化函数。HitBoost不再基于任何先验假设,但仍然具有一定的可解释性。而BecCox,主要优化Cox流派系列算法,使用单棵梯度提升树来预测发生感兴趣事件的风险比例。它在目标函数上,使用更加精确的偏似然估计函数,并且添加凸函数近似的一致性指数来缩小模型预测偏差。

    本文首先推导了自定义目标函数关于模型预测值的梯度,然后根据预测目标的类型借助XGBoost框架分别实现了HitBoost和BecCox算法。四个公开生存数据集上的实验结果显示,HitBoost相比遵循先验假设的方法以及随机生存森林,对个体生存函数的预测更加准确;BecCox相比经典的Cox比例风险模型,以及其它常用的Cox流派系列算法,也有着更好的预测性能和风险区分度。最后,本文使用来自四川大学华西医院乳腺临床研究中心的乳腺癌患者临床数据,将HitBoost方法应用于构建早期乳腺癌患者复发预后模型。凭借HitBoost在各个方面的优势,本文还使用该预后模型探究影响乳腺癌复发的重要因子以及实施治疗推荐。因此,本文提出的基于梯度提升树的优化方法可以作为有效的生存分析方法用于疾病或其它特定事件的研究及应用。

    \chinesekeyword{生存分析,机器学习,梯度提升树,比例风险模型,风险预测}
\end{chineseabstract}

