\chapter{乳腺癌复发预后模型应用}

\section{乳腺癌临床数据}
本章模型应用实验将使用来自四川大学华西医院乳腺疾病临床研究中心乳腺癌信息管理系统的真实数据。该信息管理系统回顾性收集了1989-2007年乳腺癌患者信息。自2008年起,该信息管理系统前瞻性登记所有在该院就诊的乳腺癌患者临床信息和预后信息。截止至2017年5月,乳腺癌信息管理系统总计登记患者12119例。从系统导出的原始数据包括患者的基本信息、诊断和病史、随访月经情况、病理、手术、化疗、放疗、内分泌治疗、靶向治疗记录。在华西医院就诊的乳腺癌患者,诊断日期分布在1989年到2017年之间。

本次模型应用中,我们以早期乳腺癌患者(I-III期)为研究对象,主要研究使用乳腺癌患者临床特征预测早期乳腺癌患者初次诊断后的复发情况。针对该研究目标,我们从乳腺癌信息管理系统中筛选了满足以下条件的乳腺癌患者:
\begin{enumerate}
	\item 初次诊断时,乳腺癌I-III期且无既往乳腺癌病史;
	\item 初次诊断时,仅有单侧乳腺癌。
\end{enumerate}
满足条件的患者总计11522例。从信息管理系统导出的原始数据表包括基本信息、诊断和病史、病理信息、手术、化疗、放疗、内分泌治疗、靶向治疗、随访信息。我们需要检测原始临床数据中存在的缺失、异常、重复、不一致等问题,并且采取相应措施对其进行清洗,以确保数据质量。同时,为了筛选标准的研究人群、生成并选择用于建立预测模型的临床特征,我们参考了四川大学华西医院乳腺疾病临床研究中心医学组给出的疾病专业知识、规则、建议。关于数据处理和特征筛选的工作参见文献\cite{fu2018tbme}。

最终,包含有15个乳腺癌临床特征的5293例早期乳腺癌患者记录,及其对应的末次随访记录,组成了用于本次模型应用研究的乳腺癌临床数据集 —— WCH。乳腺癌临床特征主要包括:个人基本信息、肿瘤病理信息、手术信息、化疗信息、放疗信息、内分泌治疗信息。WCH数据集的统计信息显示,在230个月的随访时间内,整个数据集中有19.82\%的患者在初次诊断后出现了复发,而随访时间的最小值、中位数、最大值分别为1、89、230。使用Kaplan-Meier方法 \citing{Kaplan1958Nonparametric}估计整个数据集人群总体的生存曲线,结果如图\ref{pic12}所示。从图\ref{pic12}中可以看出:在前200个月的随访时间内,数据集中早期乳腺癌患者的生存(即未出现复发)率是平缓下降的,但总体维持较低的复发率;在200个月以后的随访时间内,人群总体的生存率基本维持不变。

\begin{figure}[H]
\includegraphics[width=0.9\textwidth]{pic12.png}
\caption{乳腺癌临床数据人群总体生存曲线}
\label{pic12}
\end{figure}

\section{乳腺癌预后模型}

\subsection{建模过程}
为了建立针对早期乳腺癌患者的复发预后模型,我们将WCH数据集按照8:2的比例划分为了训练集和测试集。其中,训练集用于模型参数优化及模型训练,包含4234例患者记录,复发率为19.58\%;测试集用于评估模型的预测性能,包含1059例患者记录,复发率为20.77\%。同样地,我们检验了两个数据集的特征分布及生存状态的差异性,最终保证数据划分后的两个人群的特征分布无显著差异(P>0.05)。

基于第三章和第四章给出的实验结果,我们选择使用性能更加突出的HitBoost方法建立针对早期乳腺癌患者的复发预后模型。乳腺癌复发预后模型的建立过程如图\ref{pic1314}所示。在超参数优化方面,我们在WCH训练集数据上利用贝叶斯超参数优化搜索方法得到了HitBoost模型的最佳超参数,超参数搜索优化的过程如图\ref{pic13}所示。从图\ref{pic13}中可以看出,在第36轮超参数搜索时(图中红点所示),模型在交叉验证集上的表现最佳。然后,我们使用得到的模型超参数拟合WCH训练集,得到了最终的HitBoost模型。随着模型训练的进行和迭代次数的增加,模型在WCH训练集和测试集上的表现如图\ref{pic14}所示。从图\ref{pic14}可以看出,模型在WCH测试集上的表现最终趋于平稳,未出现过拟合的情况。

\begin{figure}[H]
\centering 
\subfloat[]{\label{pic13}
\includegraphics[width=0.47\textwidth]{pic13.png}
}
\hspace{0.01\linewidth}
\subfloat[]{\label{pic14}
\includegraphics[width=0.45\textwidth]{pic14.png}
}
\caption{乳腺癌复发预测模型。(a)超参数搜索过程;(b)学习曲线}
\label{pic1314}
\end{figure}

\subsection{模型结果}

经过上节所述的模型训练过程后,我们使用时间依赖的一致性指数td-CI评估模型预测性能。实验结果表明,使用HitBoost方法建立的早期乳腺癌患者复发预测模型在WCH测试集上的表现为0.72323,而在同一训练集上拟合过的CoxPH、CoxBoost、ThresReg、RSF、BecCox模型在同一测试集上的表现分别为0.69354、0.68752、0.66792、0.70517、0.71029。该实验结果进一步说明了,在缺乏先验知识或者生存数据分布未知的情况下,HitBoost生存分析方法的预测性能更加优越。这里经过WCH训练集拟合的早期乳腺癌患者复发预后模型将用于后续研究。

\begin{figure}[H]
\includegraphics[width=0.95\textwidth]{pic20.pdf}
\caption{早期乳腺癌患者复发预后模型应用流程}
\label{pic20}
\end{figure}

如图\ref{pic20}所示的模型应用流程,当乳腺癌患者完成初次诊断后,我们可以收集其个人信息和临床信息,整理得到模型规定的输入特征。然后,将该患者的特征输入至上述建立好的早期乳腺癌患者复发预后模型,经过该预后模型内部的判断、计算、决策,输出预后模型对该患者初次诊断后复发概率的估计值。最后,由模型给出的复发概率估计值,经过转化得到该患者的估计生存曲线。该曲线刻画了不同随访时长内,该患者生存(即未出现复发)的概率。最终,医生可以参考由早期乳腺癌患者复发预后模型给出的生存曲线估计,解读该患者5年内出现复发的概率,或者了解该患者的复发风险趋势,从而结合实际情况制定治疗方案或进行提前干预以减轻患者的负担。

\section{因子分析}

\subsection{重要性排序}

在完成上述早期乳腺癌患者复发预测模型的建立后,我们可以使用该预后模型来寻找对早期乳腺癌患者初次诊断后复发有重要影响的因子,或者探究不同因子的影响模式。正如第二章所描述的,复杂的神经网络模型,如DeepSurv、DeepHit、DRSA,虽然具有强大的表达能力,但却无法对模型特征做出合理解释,导致其无法用于发现与感兴趣事件相关的重要影响因子。然而,模型特征解释性在实际生存分析应用中通常是被要求的 \citing{SparanoA2016gene}。这在一定程度上限制了这类神经网络模型在各个领域中的发展与应用。因为HitBoost方法基于多输出的梯度提升树,所以它可以充分利用相关的统计量,如节点分裂过程中用于寻找最佳分裂特征的高阶梯度值,去评估特征的重要性。

\begin{figure}[H]
\includegraphics[width=0.9\textwidth]{pic15.png}
\caption{早期乳腺癌患者临床特征重要性排序}
\label{pic15}
\end{figure}

在使用预后模型评估特征重要性的过程中,我们保持模型参数不变,将HitBoost模型在WCH训练集上重复了30次模型拟合。每次都使用不同的随机数种子。最后得到30组由预测模型给出的特征重要性评估分数,将特征按分数的平均值由大到小排序,得到排在前10的特征如图\ref{pic15}所示(图中特征名为简写)。预后模型给出的对早期乳腺癌患者复发最重要的影响因子依次为诊断年龄、肿瘤N分期、分子分型、肿瘤T分期、术后化疗类型、术后内分泌治疗类型、WHO分级、Ki67、是否放疗、诊断年份。而基于英国乳腺癌患者人群构建的著名乳腺癌预后模型PREDICT\citing{Wishart2010PREDICT},经过了大规模的临床验证,并且得到了广泛的实践\citing{Candido2017PREDICT}。PREDICT模型也将诊断年龄、肿瘤T分期、肿瘤N分期、分子分型等特征作为模型输出特征用于乳腺癌复发预测,这与我们构建的乳腺癌患者预后模型得到的特征重要性评估结果基本一致。

\subsection{影响模式示例}

在评估完对早期乳腺癌患者初次诊断后复发有重要的影响因子后,我们利用预后模型探究重要影响因子的影响模式,揭示临床因素是如何影响乳腺癌复发的。肿瘤T分期指肿瘤的大小,它是乳腺疾病研究中影响乳腺癌复发最为重要的临床特征之一 \citing{AJCC8}。这里我们主要使用预后模型探究肿瘤T分期的影响模式。

\begin{figure}[H]
\includegraphics[width=0.8\textwidth]{pic16.png}
\caption{不同肿瘤T分期子组人群的实际生存曲线}
\label{pic16}
\end{figure}

在我们的研究数据中,肿瘤T分期的类别有4种,分别为T=0或1、T=2、T=3、T=4。肿瘤T分期值越大,意味着肿瘤直径越大 \citing{AJCC8}。在包含有1059例乳腺癌患者记录的WCH测试集中,按照肿瘤T分期类别划分分别得到:T=0或1的患者有373人、T=2的患者有577人、T=3的患者有73人、T=4的患者有36人。使用Kaplan-Meier方法 \citing{Kaplan1958Nonparametric}估计四个肿瘤T分期子组人群实际生存曲线,结果如图\ref{pic16}所示。图\ref{pic16}直观地说明了早期乳腺癌患者肿瘤直径越大,总体发生复发的概率也会越大。并且T=3或4的早期乳腺癌患者在前5年(60个月)内发生复发的概率是随着时间不断上升的。这与乳腺癌临床研究医学组给出的参考是一致的。

由预后模型估计,各个肿瘤T分期子组在5年内复发概率的平均值分别为0.13$\pm$0.09(T=0或1)、0.17$\pm$0.12(T=2)、0.34$\pm$0.19(T=3)、0.44$\pm$0.17(T=4)。具体地,我们展示了预后模型估计WCH测试集不同肿瘤T分期子组内患者在5年内发生复发的概率值分布,其Swarm图如图\ref{pic17:a}所示,KDE图如图\ref{pic17:b}所示。从图\ref{pic17:a}可以看出,肿瘤T分期为0/1或2的患者,在5年内发生复发的概率估计值大多都偏小,不超过0.2,且该类型患者占了较大的比重,而肿瘤分期为3或4的患者在整个测试集人群中占了较小的比重,同时5年内发生复发的概率估计值的平均值相对较高。图\ref{pic17:b}给出的概率估计值分布的KDE密度图,也直观地说明了随着乳腺癌患者肿瘤直径的增大,预后模型估计的5年内出现复发的概率值总体上也会随之变大。图中概率密度曲线重叠的区域说明了存在其它因素(如诊断年龄,分子分型等)也影响着早期乳腺癌患者的复发,且早期乳腺癌患者的复发有着复杂的影响模式。对于临床疾病研究经验中不太突出的影响因子,我们也可以利用类似的方法来探究其影响模式或发现潜在的生物标志物(biomarker),从而帮助研究者更好地认识和了解疾病,在疾病预防和控制方面制定针对性的措施。

\begin{figure}[h]
\centering 
\subfloat[]{\label{pic17:a}
\includegraphics[width=0.46\textwidth]{pic17a.png}
}
\hspace{0.01\linewidth}
\subfloat[]{\label{pic17:b}
\includegraphics[width=0.45\textwidth]{pic17b.png}
}
\caption{不同肿瘤T分期患者5年复发概率估计值的分布情况可视化结果。(a)\ Swarm图;(b)\ KDE图}
\label{pic17}
\end{figure}

\section{治疗推荐}
许多疾病预后模型一般会被用于治疗推荐,或风险量化评估。以Katzman等人 \citing{Katzman2018DeepSurv}建立的DeepSurv模型为例,它被用于构建推荐系统给来自德国乳腺癌研究人群的患者提供治疗手段推荐。结果表明遵从系统推荐的激素治疗患者人群比未遵从系统推荐的激素治疗患者人群存活时间要长,其存活时间的中位数分别为40.099个月和31.770个月。这一定程度上说明了疾病预后模型在治疗推荐上的有效性。类似地,我们基于乳腺癌患者临床真实数据构建的早期乳腺癌患者复发预后模型,也可以用于治疗推荐或量化不同治疗方式下的复发风险差异,从而为临床研究者提供一定参考。由于我们构建的乳腺癌预后模型还未处于临床验证阶段,同时随访资料数据不足以支撑实验验证,所以我们从另外一个角度来展示构建的乳腺癌预后模型如何用于治疗推荐。

以内分泌治疗和化疗为例,乳腺癌患者的内分泌治疗药物或治疗方式一般包括AI(接受芳香化酶抑制剂治疗)、SERM(接受他莫昔芬或托瑞米芬等选择性雌激素受体调节剂治疗)和OFS(接受卵巢功能抑制),而化疗药物一般包括AC(蒽环类)和TAX(紫杉类)。WCH数据集中共有1736例患者未接受内分泌治疗(No Endoc.),其余3557例患者的内分泌治疗方式包括AI、SERM、AI+SERM、OFS+AI+SERM。而未接受化疗(No Chemo.)的患者共288例,其余接受化疗的患者使用的药物包括AC+TAX、AC、TAX、其他。为了展示构建的复发预后模型在治疗推荐方面的应用,我们使用了不同的内分泌治疗和化疗组合来查看预后模型估计无复发概率的差异。

\begin{table}[H]
\caption{乳腺癌患者肿瘤分期信息及随访信息}
\begin{tabular}{ccccc}
\toprule
患者 & 肿瘤T分期 & 肿瘤N分期 & $T$(月) & $\delta$ \\ 
\midrule
A & 1 & 3 & 75 & 0 \\
B & 1 & 0 & 62 & 0 \\
C & 1 & 0 & 86 & 0 \\
D & 2 & 2 & 34 & 1 \\
\bottomrule
\end{tabular}
\label{table06}
\end{table}

首先,我们从WCH测试集中随机选择了4例乳腺癌患者,分别记为A、B、C、D,他们的肿瘤分期信息和随访信息如表\ref{table06}所示。一般来说,肿瘤分期数值越高,表示肿瘤进展程度越高\citing{AJCC8}。然后,我们尝试对这些患者使用不同的内分泌治疗方式,并且分别对比查看未接受化疗和接受AC+TAX化疗情况下,预后模型对其估计的5年无复发概率。实验结果如图\ref{pic18}所示。以患者A为例,从图中我们可以看出,当该患者采用OFS+AI+SERM内分泌治疗时,相比未接受化疗而言,接受AC+TAX化疗带来的5年无复发概率估计提高了3个百分点。另外,无论该患者是否接受化疗,接受任何类型的内分泌治疗相比未接受内分泌治疗来说,其5年无复发概率估计均有接近8个百分点的提升。以上的对比情况对于患者B和C来说5年无复发概率差异并不明显,这在一定程度上是因为他们的肿瘤进展程度相对较低(T分期为1,N分期为0)。而患者D表现出和患者A相似的5年无复发概率差异,他们的肿瘤进展程度都相对较高。通过以上的分析及展示,我们可以知道:乳腺疾病临床医生或研究者,可以通过复发预后模型更加科学地评估不同治疗方式带来的影响,从而结合临床实际情况制定更好的治疗措施来减轻患者的负担。

\begin{figure}[H]
\includegraphics[width=0.95\textwidth]{pic18.png}
\caption{不同内分泌治疗和化疗方式组合下患者的5年无复发概率}
\label{pic18}
\end{figure}

\section{本章小结}

本章主要介绍了我们提出的生存分析优化方法在乳腺癌临床研究领域的实际应用。本章首先介绍了乳腺癌临床数据的基本情况,包括其基本统计信息、生存曲线、数据处理流程等。然后,我们将表现较好的HitBoost生存分析方法应用于建立针对早期乳腺癌患者的复发预测模型。在乳腺癌临床数据集上训练完复发预测模型后,我们重点讨论了模型在因子分析、治疗推荐方面的应用。因为HitBoost方法具有一定的模型解释性,所以在因子分析方面,我们借助预后模型直接评估对乳腺癌复发有重要影响的因子,得到的结果与PREDICT模型纳入特征基本一致。同时,我们还借助预后模型探究了肿瘤T分期对乳腺癌复发的影响模式。另外,在治疗推荐方面,我们以乳腺癌患者内分泌治疗+化疗推荐为例,展示了使用预后模型来帮助治疗推荐和分析的过程。

实验结果显示,我们建立的早期乳腺癌患者初次诊断后复发预后模型,可以用于乳腺临床疾病生存研究或辅助诊疗。本文提出的基于梯度提升树的生存分析优化方法,综合考虑了模型预测性能和实用性。因此,它可以作为一种有效的方法用于各个领域的生存分析或预测。
