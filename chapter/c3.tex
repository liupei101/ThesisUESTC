
\chapter{HitBoost生存分析方法}

针对现有生存分析方法依赖先验假设或解释性不足的问题,我们在传统的FHT首次命中时间模型的基础上研究了基于梯度提升树的优化方法,提出一种新的生存分析方法:HitBoost。该方法不再依赖任何先验假设,使用多输出的梯度提升树直接预测首次命中时间的概率分布,同时保证了一定的模型解释性。考虑到XGBoost梯度提升树算法在模型表征能力以及避免过拟合等方面的优势,我们在该框架的基础上研究并实现了HitBoost生存分析方法。下面将首先从模型原理,目标函数及梯度计算三个方面介绍提出的HitBoost生存分析方法,然后介绍四个公开的生存数据集的基本情况,最后给出HitBoost方法在公开数据集上的实验结果。

\section{HitBoost}

\subsection{模型介绍}

为了不再依赖个体风险函数的任何假设,直接学习协变量与首次发生事件时间概率分布之间潜在的关联,同时保证模型的特征解释性,我们提出了一种基于多输出的梯度提升树的生存分析方法——HitBoost。如图\ref{pic01}所示,HitBoost模型是一个多输出的梯度提升树模型,它学习并且直接估计在给定协变量$x^*$的条件下,在$t^*$首次发生事件的概率$P(t = t^*,\delta = 1\mid x = x^* )$。HitBoost模型的输入为协变量(或生存数据),每一个输出由一个单独的梯度提升树模型给出。作为一个前向加法模型,每个单独的梯度提升树模型由很多棵决策树组成。多个独立的梯度提升树模型经过一个softmax层转化为最终的输出预测值。模型预测值$\hat{y}$是一个向量,写作:
\begin{equation}
\hat{y}=\left[\hat{y}_1, \hat{y}_1, \dots, \hat{y}_{T_{max}}, \hat{y}_{T_{max}+1}\right] \label{F31}
\end{equation}

其中$T_{max}$示研究对象中个体最长随访时间。给定某个个体协变量$x$,模型的一个输出$\hat{y}_t$表示该个体在$t$时刻发生事件的预测概率值$\hat{P}(t, \delta = 1 \mid x)$。因为$\hat{y}_{T_{max}+1}=1-\sum_{t=1}^{T_{max}} \hat{y}_t$,所以预测值$\hat{P}(T_{max}+1, \delta = 1 \mid x)$表示了协变量为$x$的个体在整个研究过程中未观测到事件发生。这里,我们定义个体$i$的累积事件函数CIF(Cumulative Incidence Function)来表示个体$i$在$t$时刻或$t$时刻之前发生事件的累积概率,其表达式如下:
\begin{equation}
\hat{F}(i, t) = P(\tau \le t, \delta = 1 \mid x_i) = \sum_{\tau \le t} \hat{y}_{\tau}^i \label{F32}
\end{equation}

\begin{figure}[h]
\includegraphics[width=\textwidth]{pic01.png}
\caption{HitBoost模型}
\label{pic01}
\end{figure}

在给定生存数据和特定的目标函数(见下节)后,多颗梯度提升树模型在每一轮迭代的过程中通过梯度提升算法并行地学习协变量和风险函数之间的潜在关系。最后,经过拟合的HitBoost生存分析模型可以准确地预测首次发生事件时间的概率密度函数,也可以经过转换得到风险函数和生存函数的估计值。HitBoost方法可以学习复杂的风险函数表达,而不仅限于比例风险函数或形式固定的随机过程。而且,由于该方法基于多输出的梯度提升树,所以它可以充分利用相关的统计量,如节点分裂过程中用于寻找最佳分裂特征的高阶梯度值,去探索和感兴趣事件相关的重要特征,这也是HitBoost方法的优点之一。

\subsection{目标函数}

为了训练生存分析模型,我们需要定义特定的目标函数的表达式。HitBoost生存分析方法需要优化的目标函数包括两个:
\begin{enumerate}
    \item $L_1$,由公式给出的在首次命中时间模型中用到的极大似然估计函数;
    \item $L_2$,凸函数近似的一致性指数,它作为损失函数的一部分用于调整风险排序。
\end{enumerate}
所以,最终HitBoost方法需要最小化的目标函数是:
\begin{equation}
L=\theta \cdot L_1 + (1-\theta) \cdot L_2 \label{F33}
\end{equation}
其中$\theta \in \mathbb{R}$且满足 $0 \le \theta \le 1$,是模型需要调整的超参数之一。
 
由第二章第二节给出的表达式可知,$L_1$的表达式如下:$$
L_1 = -\sum_{i=1}^{n} \left[ I(\delta_i=1)\cdot ln(\hat{y}_{T_i}^i) + I(\delta_i=0)\cdot ln(1-\sum_{t\le T_i}\hat{y}_t^i) \right]
$$ 
对于个体$i$,$L_1$项保证了若该个体在$T_i$时刻发生了事件,则模型最大化其估计的在$T_i$时刻首次发生事件的概率;若该个体在$T_i$时刻出现失访(未观测到时间发生),则模型最小化其估计的在$T_i$时刻或$T_i$时刻之前首次发生事件的概率。

对于目标函数$L_2$,一致性指数被当作了优化项。一般来说,一致性指数作为生存预测模型评估指标出现在模型性能评估中,认为发生事件时间更早的个体所对应的模型风险估计值也应该更高。所以,一致性指数中风险排序的思想也被应用到生存分析模型的目标函数中。但是,由于一致性指数不是一个凸函数,所以它无法被直接引入作为梯度提升树的优化目标。从\cite{Yan2004pre}得到启发,我们使用了凸函数$$
\phi(x,y)=
\begin{cases}
{[-(x-y-\gamma)]}^n & \text{if } x-y < \gamma,\\
0 & \text{if } x-y \ge \gamma.
\end{cases}
$$
去近似一致性指数中的指示函数,而不是熟悉的sigmoid函数。凸函数$\phi$中,模型超参数满足$0<\gamma \le 1$和$n>1$。对一致性指数进行调整,并且使用$\hat{F}(i, T_i)$代替个体$i$的预测风险后,可以得到$$
L_2 = \frac{\sum_{(i,j)\in \Omega} W_{i,j}\cdot \phi(\hat{F}(i, T_i), \hat{F}(j, T_i))}{\sum_{(i,j)\in \Omega} W_{i,j}}
$$
其中,分母$$
W_{i,j}=-\left( \hat{F}(i, T_i) - \hat{F}(j, T_i) \right)
$$
为一个归一化系数。它是集合$\Omega$中每个元组$(i,j)$的权重系数,表示个体$i$和$j$之间的风险差值。使用凸函数$\phi$近似后的一致性指数$L_2$的作用主要表现在以下两个方面:
\begin{enumerate}
    \item 如果个体$i$比个体$j$更早发生事件,也就是说个体$i$有更高的事件发生风险,最小化$L_2$等于加大个体$i$和$j$之间的风险差值,直到这个差值大于或等于$\gamma$;
    \item 如果集合$\Omega$中某个元组$(i,j)$的风险差异大于或等于$\gamma$,那么这个元组将不会对目标函数$L_2$的大小产生任何影响。
\end{enumerate}
这种机制可以有效避免模型训练过程中的过拟合 \citing{Yan2003opt}。

\subsection{梯度计算}

在易拓展,灵活的梯度提升树框架XGBoost的帮助下,我们实现了提出的HitBoost生存分析方法。不同于一般的梯度提升树模型,XGBoost需要推导目标函数关于模型预测值的一阶梯度和二阶梯度,而一般的梯度提升树模型不需要二阶梯度信息。这里,我们以定理的形式直接给出梯度推导的结果。定理的证明将在附录中给出。

\begin{theorem}\label{thm:1.1}
对观测时间和观测事件状态分别为$\delta_k$和$T_k$的个体$k$,HitBoost目标函数$L_1$关于模型预测值$\hat{y}_t^k$的\textbf{一阶梯度}为$$
\frac{\partial L_1}{\partial \hat{y}_t^k}=
\begin{cases}
I(t=T_k)\cdot \frac{-1}{\hat{y}_t^k} & \text{if } \delta_k = 1,\\
I(t\le T_k)\cdot \frac{1}{1-\hat{F}(k, T_k)} & \text{if } \delta_k = 0.
\end{cases}
$$
\end{theorem}

\begin{theorem}\label{thm:1.2}
对观测时间和观测事件状态分别为$\delta_k$和$T_k$的个体$k$,HitBoost目标函数$L_1$关于模型预测值$\hat{y}_t^k$的\textbf{二阶梯度}为$$
\frac{\partial^2 L_1}{\partial \hat{y}_t^k}=
\begin{cases}
I(t=T_k)\cdot \frac{-1}{{(\hat{y}_t^k)}^2} & \text{if } \delta_k = 1,\\
I(t\le T_k)\cdot \frac{1}{{[1-\hat{F}(k, T_k)]}^2} & \text{if } \delta_k = 0.
\end{cases}
$$
\end{theorem}

在给出关于目标函数$L_2$的梯度的定理前,我们首先介绍一些符号约定。我们定义和个体$k$有关的两个互不相交子集$$
\Omega_1=\{(k,i) \mid \delta_k=1,T_k < T_i\}
$$
和$$
\Omega_2=\{(i,k) \mid \delta_i=1,T_i < T_k\}
$$
分别表示个体$k$的发生事件风险大于个体$i$和小于个体$i$的元组集合。设$L_2$的分母和分子分别为$\alpha$和$\beta$,即\[
\begin{split}
\alpha &= \sum_{(i,j)\in \Omega} W_{i,j}\\
\beta &= \sum_{(i,j)\in \Omega} W_{i,j} \cdot \phi\left[ \hat{F}(i, T_i), \hat{F}(j, T_i) \right]
\end{split}
\]
我们有如下定理。

\begin{theorem}\label{thm:1.3}
对观测时间和观测事件状态分别为$\delta_k$和$T_k$的个体$k$,HitBoost目标函数$L_2$项分母$\alpha$关于模型预测值$\hat{y}_t^k$的\textbf{一阶梯度}为$$
\frac{\partial \alpha}{\partial \hat{y}_t^k}=\alpha^{'}
\begin{cases}
I(t\le T_k)\cdot {\sum\limits_{i: T_i>T_k}(-1)} + \sum\limits_{i: \delta_i=1,T_i<T_k} I(t\le T_i) & \text{if } \delta_k = 1,\\
\sum\limits_{i: \delta_i=1,T_i<T_k} I(t\le T_i) & \text{if } \delta_k = 0.
\end{cases}
$$ 且$L_2$项分子$\beta$关于模型预测值$\hat{y}_t^k$的\textbf{一阶梯度}为$$
\frac{\partial \beta}{\partial \hat{y}_t^k}=\beta^{'}
\begin{cases}
\frac{\partial \beta}{\partial \hat{y}_t^k} \mid_{\Omega_1} + \frac{\partial \beta}{\partial \hat{y}_t^k} \mid_{\Omega_2} & \text{if } \delta_k = 1,\\
\frac{\partial \beta}{\partial \hat{y}_t^k} \mid_{\Omega_2} & \text{if } \delta_k = 0.
\end{cases}
$$ 其中\[
\begin{split}
\frac{\partial \beta}{\partial \hat{y}_t^k} \mid_{\Omega_1} &= I(t\le T_k)\cdot \sum\limits_{(k,i)\in \Omega_1} {I(-W_{k,i}<\gamma)\cdot (W_{k,i}+\gamma)^{n-1}\cdot [-(n+1)\cdot W_{k,i}-\gamma]} \\
\frac{\partial \beta}{\partial \hat{y}_t^k} \mid_{\Omega_2} &= \sum\limits_{(i,k)\in \Omega_2} {I(t\le T_i)\cdot I(-W_{i,k}<\gamma)\cdot (W_{i,k}+\gamma)^{n-1}\cdot [(n+1)\cdot W_{i,k}+\gamma]}
\end{split}
\]
\end{theorem}

\begin{theorem}\label{thm:1.4}
对观测时间和观测事件状态分别为$\delta_k$和$T_k$的个体$k$,HitBoost目标函数$L_2$项分母$\alpha$关于模型预测值$\hat{y}_t^k$的\textbf{二阶梯度}为$$
\frac{\partial^2 \alpha}{\partial \hat{y}_t^k}=\alpha^{''}=0
$$ 且$L_2$项分子$\beta$关于模型预测值$\hat{y}_t^k$的\textbf{二阶梯度}为$$
\frac{\partial^2 \beta}{\partial \hat{y}_t^k}=\beta^{''}=
\begin{cases}
\frac{\partial^2 \beta}{\partial \hat{y}_t^k} \mid_{\Omega_1} + \frac{\partial^2 \beta}{\partial \hat{y}_t^k} \mid_{\Omega_2} & \text{if } \delta_k = 1,\\
\frac{\partial^2 \beta}{\partial \hat{y}_t^k} \mid_{\Omega_2} & \text{if } \delta_k = 0.
\end{cases}
$$ 其中\[
\begin{split}
\frac{\partial^2 \beta}{\partial \hat{y}_t^k} \mid_{\Omega_1} =& I(t\le T_k)\cdot \sum\limits_{(k,i)\in \Omega_1} I(-W_{k,i}<\gamma)\cdot \\
  & \left\{(n+1)\cdot (W_{k,i}+\gamma)^{n-1} + (n-1)\cdot (W_{k,i}+\gamma)^{n-2}\cdot [(n+1)\cdot W_{k,i}+\gamma]\right\} \\
\frac{\partial^2 \beta}{\partial \hat{y}_t^k} \mid_{\Omega_2} =& \sum\limits_{(i,k)\in \Omega_2} I(t\le T_i)\cdot I(-W_{i,k}<\gamma)\cdot \\
  & \left\{(n+1)\cdot (W_{i,k}+\gamma)^{n-1} + (n-1)\cdot (W_{i,k}+\gamma)^{n-2}\cdot [(n+1)\cdot W_{i,k}+\gamma]\right\}
\end{split}
\]
\end{theorem}

由定理\ref{thm:1.1},\ref{thm:1.2},\ref{thm:1.3}和\ref{thm:1.4},以及$$
L=\theta\cdot L_1 + (1-\theta)\cdot L_2= \theta\cdot L_1 + (1-\theta)\cdot \frac{\beta}{\alpha}
$$ 我们可以使用链式法则轻松推导出HitBoost目标函数$L$关于模型预测值$\hat{y}_t^k$的\textbf{一阶梯度}$$
\frac{\partial L}{\partial \hat{y}_t^k}=\theta\cdot \frac{\partial L_1}{\partial \hat{y}_t^k} + (1-\theta)\cdot \omega(\alpha, \beta)
$$ 其中$\omega(\alpha, \beta)=\frac{\beta^{'}\cdot \alpha - \beta\cdot \alpha^{'}}{\alpha^2}$。HitBoost目标函数$L$关于模型预测值$\hat{y}_t^k$的\textbf{二阶梯度}$$
\frac{\partial^2 L}{\partial \hat{y}_t^k}=\theta\cdot \frac{\partial^2 L_1}{\partial \hat{y}_t^k} + (1-\theta)\cdot \tau(\alpha, \beta)
$$ 其中$\tau(\alpha, \beta)=\frac{\alpha\cdot (\beta^{''}\cdot \alpha - \beta\cdot \alpha^{''})-2\alpha^{'}\cdot (\beta^{'}\cdot \alpha - \beta\cdot \alpha^{'})}{\alpha^3}$。

在实现HitBoost生存分析方法的过程中,我们在计算目标函数梯度部分使用了向量化技巧,显著提高了计算效率且缩短了模型拟合时间。具体实现可以参考源代码https://github.com/liupei101/libsurv。

\section{实验生存数据集}

在实验评估提出生存分析方法之前,我们首先介绍实验将会用到的四个公开生存数据集。四个公开生存数据集都来源于健康医疗领域,但它们所研究的感兴趣的事件不尽相同。在收集数据进行预处理的过程中,我们统一以月份(30天)为观测时间的基本单位。为了了解生存数据集的基本情况,我们从研究事件,样本量,数据特征数目,事件占比,观测时间最小值,观测时间中位数和观测时间最大值几个方面观察了数据,统计结果如下表1。Kaplan-Meier方法对各个数据集的估计生存曲线如图2所示,可以直观看出SUPPORT数据集中总体的生存状况相对而言是最差的,同时观测时间最短,而METABRIC数据集中总体的生存状态最好,同时观测时间最长。

\subsection{WHAS}

\subsection{SUPPORT}

\subsection{METABRIC}

\subsection{ROTT2}

\section{实验结果}

\subsection{实验设置}

\subsection{模型性能}

\subsection{样例分析}

\section{本章小结}

本章首先从原理,目标函数和实现三个方面介绍了提出的HitBoost生存分析方法。简单来说,该方法基于传统的FHT首次命中时间模型,使用梯度提升树实现。具体地,HitBoost方法使用多输出的梯度提升树直接预测感兴趣的事件发生随时间的概率分布,并且把生存分析相关的极大似然估计函数作为目标函数,同时还另外加入凸函数近似的一致性指数来调整风险排序。一方面它不再对个体风险函数的形式做出任何假设,另一方面它作为决策树的拓展模型仍然具有一定的解释性。在模型实现上,我们首先推导了自定义的目标函数关于模型预测值的一阶梯度和二阶梯度,然后根据预测目标的类型借助灵活且易拓展的XGBoost梯度提升树框架实现了提出的HitBoost生存分析方法。

然后,本章介绍了四个公开的生存数据集:WHAS,SUPPORT,METABRIC,ROTT2。主要从常用统计量和人群生存曲线的角度给出了数据集的基本情况。最后,本章给出了HitBoost方法在公开数据集上的实验结果。HitBoost预测模型在独立测试集上的预测性能以及对个体生存函数的估计结果显示,我们提出的基于梯度提升树的优化方法相比一类遵循风险假设的模型,以及随机生存森林来说,具有更好的预测性能及风险区分度。
